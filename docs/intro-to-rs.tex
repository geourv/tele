% Options for packages loaded elsewhere
\PassOptionsToPackage{unicode}{hyperref}
\PassOptionsToPackage{hyphens}{url}
%
\documentclass[
]{book}
\usepackage{lmodern}
\usepackage{amssymb,amsmath}
\usepackage{ifxetex,ifluatex}
\ifnum 0\ifxetex 1\fi\ifluatex 1\fi=0 % if pdftex
  \usepackage[T1]{fontenc}
  \usepackage[utf8]{inputenc}
  \usepackage{textcomp} % provide euro and other symbols
\else % if luatex or xetex
  \usepackage{unicode-math}
  \defaultfontfeatures{Scale=MatchLowercase}
  \defaultfontfeatures[\rmfamily]{Ligatures=TeX,Scale=1}
\fi
% Use upquote if available, for straight quotes in verbatim environments
\IfFileExists{upquote.sty}{\usepackage{upquote}}{}
\IfFileExists{microtype.sty}{% use microtype if available
  \usepackage[]{microtype}
  \UseMicrotypeSet[protrusion]{basicmath} % disable protrusion for tt fonts
}{}
\makeatletter
\@ifundefined{KOMAClassName}{% if non-KOMA class
  \IfFileExists{parskip.sty}{%
    \usepackage{parskip}
  }{% else
    \setlength{\parindent}{0pt}
    \setlength{\parskip}{6pt plus 2pt minus 1pt}}
}{% if KOMA class
  \KOMAoptions{parskip=half}}
\makeatother
\usepackage{xcolor}
\IfFileExists{xurl.sty}{\usepackage{xurl}}{} % add URL line breaks if available
\IfFileExists{bookmark.sty}{\usepackage{bookmark}}{\usepackage{hyperref}}
\hypersetup{
  pdftitle={Teledetecció per a la planificació territorial},
  pdfauthor={Benito Zaragozí},
  hidelinks,
  pdfcreator={LaTeX via pandoc}}
\urlstyle{same} % disable monospaced font for URLs
\usepackage{color}
\usepackage{fancyvrb}
\newcommand{\VerbBar}{|}
\newcommand{\VERB}{\Verb[commandchars=\\\{\}]}
\DefineVerbatimEnvironment{Highlighting}{Verbatim}{commandchars=\\\{\}}
% Add ',fontsize=\small' for more characters per line
\usepackage{framed}
\definecolor{shadecolor}{RGB}{248,248,248}
\newenvironment{Shaded}{\begin{snugshade}}{\end{snugshade}}
\newcommand{\AlertTok}[1]{\textcolor[rgb]{0.94,0.16,0.16}{#1}}
\newcommand{\AnnotationTok}[1]{\textcolor[rgb]{0.56,0.35,0.01}{\textbf{\textit{#1}}}}
\newcommand{\AttributeTok}[1]{\textcolor[rgb]{0.77,0.63,0.00}{#1}}
\newcommand{\BaseNTok}[1]{\textcolor[rgb]{0.00,0.00,0.81}{#1}}
\newcommand{\BuiltInTok}[1]{#1}
\newcommand{\CharTok}[1]{\textcolor[rgb]{0.31,0.60,0.02}{#1}}
\newcommand{\CommentTok}[1]{\textcolor[rgb]{0.56,0.35,0.01}{\textit{#1}}}
\newcommand{\CommentVarTok}[1]{\textcolor[rgb]{0.56,0.35,0.01}{\textbf{\textit{#1}}}}
\newcommand{\ConstantTok}[1]{\textcolor[rgb]{0.00,0.00,0.00}{#1}}
\newcommand{\ControlFlowTok}[1]{\textcolor[rgb]{0.13,0.29,0.53}{\textbf{#1}}}
\newcommand{\DataTypeTok}[1]{\textcolor[rgb]{0.13,0.29,0.53}{#1}}
\newcommand{\DecValTok}[1]{\textcolor[rgb]{0.00,0.00,0.81}{#1}}
\newcommand{\DocumentationTok}[1]{\textcolor[rgb]{0.56,0.35,0.01}{\textbf{\textit{#1}}}}
\newcommand{\ErrorTok}[1]{\textcolor[rgb]{0.64,0.00,0.00}{\textbf{#1}}}
\newcommand{\ExtensionTok}[1]{#1}
\newcommand{\FloatTok}[1]{\textcolor[rgb]{0.00,0.00,0.81}{#1}}
\newcommand{\FunctionTok}[1]{\textcolor[rgb]{0.00,0.00,0.00}{#1}}
\newcommand{\ImportTok}[1]{#1}
\newcommand{\InformationTok}[1]{\textcolor[rgb]{0.56,0.35,0.01}{\textbf{\textit{#1}}}}
\newcommand{\KeywordTok}[1]{\textcolor[rgb]{0.13,0.29,0.53}{\textbf{#1}}}
\newcommand{\NormalTok}[1]{#1}
\newcommand{\OperatorTok}[1]{\textcolor[rgb]{0.81,0.36,0.00}{\textbf{#1}}}
\newcommand{\OtherTok}[1]{\textcolor[rgb]{0.56,0.35,0.01}{#1}}
\newcommand{\PreprocessorTok}[1]{\textcolor[rgb]{0.56,0.35,0.01}{\textit{#1}}}
\newcommand{\RegionMarkerTok}[1]{#1}
\newcommand{\SpecialCharTok}[1]{\textcolor[rgb]{0.00,0.00,0.00}{#1}}
\newcommand{\SpecialStringTok}[1]{\textcolor[rgb]{0.31,0.60,0.02}{#1}}
\newcommand{\StringTok}[1]{\textcolor[rgb]{0.31,0.60,0.02}{#1}}
\newcommand{\VariableTok}[1]{\textcolor[rgb]{0.00,0.00,0.00}{#1}}
\newcommand{\VerbatimStringTok}[1]{\textcolor[rgb]{0.31,0.60,0.02}{#1}}
\newcommand{\WarningTok}[1]{\textcolor[rgb]{0.56,0.35,0.01}{\textbf{\textit{#1}}}}
\usepackage{longtable,booktabs}
% Correct order of tables after \paragraph or \subparagraph
\usepackage{etoolbox}
\makeatletter
\patchcmd\longtable{\par}{\if@noskipsec\mbox{}\fi\par}{}{}
\makeatother
% Allow footnotes in longtable head/foot
\IfFileExists{footnotehyper.sty}{\usepackage{footnotehyper}}{\usepackage{footnote}}
\makesavenoteenv{longtable}
\usepackage{graphicx,grffile}
\makeatletter
\def\maxwidth{\ifdim\Gin@nat@width>\linewidth\linewidth\else\Gin@nat@width\fi}
\def\maxheight{\ifdim\Gin@nat@height>\textheight\textheight\else\Gin@nat@height\fi}
\makeatother
% Scale images if necessary, so that they will not overflow the page
% margins by default, and it is still possible to overwrite the defaults
% using explicit options in \includegraphics[width, height, ...]{}
\setkeys{Gin}{width=\maxwidth,height=\maxheight,keepaspectratio}
% Set default figure placement to htbp
\makeatletter
\def\fps@figure{htbp}
\makeatother
\setlength{\emergencystretch}{3em} % prevent overfull lines
\providecommand{\tightlist}{%
  \setlength{\itemsep}{0pt}\setlength{\parskip}{0pt}}
\setcounter{secnumdepth}{5}
\usepackage{booktabs}

\ifxetex
  \usepackage{polyglossia}
  \setmainlanguage{catalan}
  % Tabla en lugar de cuadro
  \gappto\captionsspanish{\renewcommand{\tablename}{Taula}  
          \renewcommand{\listtablename}{Índex de taules}}
\else
  \usepackage[catalan]{babel}
\fi
\usepackage[]{natbib}
\bibliographystyle{apalike}

\title{Teledetecció per a la planificació territorial}
\author{Benito Zaragozí}
\date{28-Feb-2024}

\begin{document}
\maketitle

{
\setcounter{tocdepth}{1}
\tableofcontents
}
\hypertarget{presentaciuxf3-de-lassignatura}{%
\chapter*{Presentació de l'Assignatura}\label{presentaciuxf3-de-lassignatura}}
\addcontentsline{toc}{chapter}{Presentació de l'Assignatura}

La teledetecció, una disciplina que es basa en l'adquisició i anàlisi de dades sobre la Terra des de sensors situats en plataformes espacials o aerotransportades, s'ha convertit en una eina indispensable en l'estudi i gestió del nostre planeta. L'objectiu d'aquesta assignatura és proporcionar als estudiants una comprensió profunda dels principis teòrics i pràctics de la teledetecció, així com les seves aplicacions en diversos camps com l'agricultura, la gestió de recursos naturals, l'urbanisme, i la vigilància ambiental.
Objectius de l'Assignatura

L'assignatura pretén dotar l'estudiantat de:

\begin{itemize}
\tightlist
\item
  \textbf{Coneixements Fonamentals:} Entendre els principis físics que regeixen la teledetecció, incloent la interacció de la radiació electromagnètica amb la matèria.
\item
  \textbf{Habilitats Tècniques:} Adquirir destresa en el maneig de dades de teledetecció, incloent l'adquisició, processament, i anàlisi d'imatges satèl·lit i dades aerotransportades.
\item
  \textbf{Aplicació Pràctica:} Ser capaç d'aplicar coneixements i habilitats tècniques a l'estudi de casos reals, mitjançant l'ús de programari específic com QGIS i eines de processament d'imatges.
\end{itemize}

\hypertarget{requeriments}{%
\section*{Requeriments}\label{requeriments}}
\addcontentsline{toc}{section}{Requeriments}

Es recomana que els estudiants tinguin un coneixement bàsic de SIG (Sistemes d'Informació Geogràfica) i estiguin familiaritzats amb conceptes bàsics de geografia i paisatge. A més, es valorarà la capacitat d'autogestió i investigació autònoma (bibliografia recomanada, cerques per internet, IA generatives, tutories, etc).

\hypertarget{programari-necessari}{%
\subsection*{Programari Necessari}\label{programari-necessari}}
\addcontentsline{toc}{subsection}{Programari Necessari}

Per seguir les explicacions d'aquest curs serà necessari instal·lar el següent programari:

\begin{itemize}
\tightlist
\item
  QGIS +3.14 (software lliure): Amb un llistat de connectors que s'instalaran al llarg del curs. QGIS és un sistema d'informació geogràfica que permet la visualització, l'edició i l'anàlisi de dades geoespacials.
\item
  R i RStudio: R és un entorn i llenguatge de programació amb enfocament a l'estadística. RStudio proporciona una interfície amigable per a l'ús de R. Serà necessari instal·lar els paquets knitr i rmarkdown, essencials per a la creació de documents i informes dinàmics.
\end{itemize}

Configuració Recomanada per R

\begin{Shaded}
\begin{Highlighting}[]
\KeywordTok{sessionInfo}\NormalTok{()}
\end{Highlighting}
\end{Shaded}

Aquest codi R mostra la informació de la sessió actual de R, incloent la versió de R, el sistema operatiu, i els paquets carregats. Això pot ajudar a assegurar que tots els estudiants treballin amb una configuració similar i compatible amb el material del curs.

\hypertarget{altres-recomanacions}{%
\subsection*{Altres Recomanacions}\label{altres-recomanacions}}
\addcontentsline{toc}{subsection}{Altres Recomanacions}

\begin{itemize}
\tightlist
\item
  7zip: Programa de compressió recomanat per a descomprimir els materials del curs.
\end{itemize}

Es recomana revisar la compatibilitat del programari amb el sistema operatiu de l'ordinador personal de cada estudiant i realitzar les instal·lacions amb antelació per assegurar un inici de curs sense inconvenients.

\hypertarget{estructura-de-lassignatura}{%
\subsection*{Estructura de l'Assignatura}\label{estructura-de-lassignatura}}
\addcontentsline{toc}{subsection}{Estructura de l'Assignatura}

L'assignatura està dividida en diverses seccions que cobreixen des de la introducció a la teledetecció, els principis físics subjacents, les plataformes i sensors, fins a les tècniques de processament d'imatges i les seves aplicacions. A més, s'inclourà una secció especial sobre l'ús de drones per a la teledetecció, considerant el context i la normativa que afecten Espanya.

\hypertarget{metodologia}{%
\subsection*{Metodologia}\label{metodologia}}
\addcontentsline{toc}{subsection}{Metodologia}

L'ensenyament combinarà classes teòriques amb sessions pràctiques a ``l'Aula de SIG''. Les classes teòriques proporcionaran els coneixements necessaris per entendre els fonaments de la teledetecció, mentre que les pràctiques permetran als estudiants aplicar aquests coneixements a situacions reals, fent ús de dades de teledetecció i eines de processament d'imatges. A les classes teòriques farem emfasi en els coneixements més importants i facilitarem exemples clars.

\hypertarget{avaluaciuxf3}{%
\subsection*{Avaluació}\label{avaluaciuxf3}}
\addcontentsline{toc}{subsection}{Avaluació}

L'avaluació de l'assignatura serà contínua, basant-se en la participació en classe, la realització de pràctiques, i la presentació de projectes. A més, hi haurà exàmens parcials per avaluar els coneixements adquirits en les seccions teòriques de l'assignatura. Aquests apunts, o una versió millorada, podran ser portats als examens escrits.

\hypertarget{exercicis}{%
\section{Exercicis}\label{exercicis}}

\begin{itemize}
\tightlist
\item
  Utilizando Rstudio, crea tu primer documento con Rmarkdown. El documento debe mostrar los metadatos básicos (título, autor y fecha), un título de primer nivel (por ejemplo, `Información de la sesión') y un recuadro con la información básica de vuestra sesión de R.
\end{itemize}

\hypertarget{intro}{%
\chapter{Introducció a la Teledetecció}\label{intro}}

La teledetecció, un terme que es va originar als anys 1960, designa el procés d'observar i obtenir informació sobre la Terra des de la distància, sense estar en contacte físic directe amb l'àrea d'estudi. Aquesta disciplina utilitza sensors instal·lats en plataformes aèries o espacials per captar i registrar la radiació electromagnètica reflectida o emesa per la superfície terrestre i els objectes presents en ella.

\hypertarget{principis-fonamentals}{%
\section{Principis Fonamentals}\label{principis-fonamentals}}

La teledetecció es basa en el principi que diferents objectes i materials a la superfície de la Terra reflecteixen i absorbeixen la llum solar de manera diferent. Aquestes diferències en la reflectància i l'emissivitat es poden detectar i mesurar amb sensors, permetent la identificació i l'anàlisi d'aquests objectes a grans distàncies.

\hypertarget{aplicacions}{%
\section{Aplicacions}\label{aplicacions}}

La teledetecció ofereix als geògrafs una eina potent per a l'anàlisi i comprensió del paisatge terrestre, els seus processos naturals i les activitats humanes que impacten en ell. A continuació, es detalla com la teledetecció es relaciona amb diverses disciplines dins de la geografia i com es pot utilitzar per enriquir l'estudi d'aquestes àrees:

\hypertarget{biogeografia}{%
\subsection{Biogeografia}\label{biogeografia}}

En biogeografia, la teledetecció permet el monitoratge de la distribució i la dinàmica de la vegetació, els hàbitats i les espècies a través de grans àrees i períodes de temps. Les imatges de satèl·lit ajuden a identificar els patrons de biodiversitat, els canvis en la coberta vegetal i els impactes del canvi climàtic i les activitats humanes sobre els ecosistemes naturals.

\hypertarget{cartografia}{%
\subsection{Cartografia}\label{cartografia}}

La teledetecció és fonamental per a la cartografia moderna, proporcionant dades precises i actualitzades per a la creació i actualització de mapes. Les imatges de satèl·lit i aèries són utilitzades per a mapejar característiques físiques i humanes, inclòs el relleu, l'ús del sòl i la infraestructura urbana, facilitant així la planificació territorial i la gestió de recursos.

\hypertarget{geografia-urbana}{%
\subsection{Geografia Urbana}\label{geografia-urbana}}

En el camp de la geografia urbana, la teledetecció permet analitzar el creixement urbà, la morfologia de les ciutats i els patrons d'ús del sòl urbà. Les dades obtingudes ajuden a entendre millor les dinàmiques urbanes, inclòs l'esprai urbà, la densitat de població i la distribució d'espais verds, així com a identificar àrees de degradació ambiental i planificar el desenvolupament urbà sostenible.

\hypertarget{climatologia}{%
\subsection{Climatologia}\label{climatologia}}

La teledetecció juga un paper clau en la climatologia, oferint dades essencials per a l'estudi dels patrons climàtics, el seguiment de fenòmens meteorològics extrems i la investigació del canvi climàtic. Les observacions des de l'espai proporcionen informació valuosa sobre la temperatura de la superfície, els patrons de precipitació, la cobertura de núvols i els canvis en els glaciars i el gel marí.

\hypertarget{altres-uxe0rees}{%
\subsection{Altres àrees}\label{altres-uxe0rees}}

La teledetecció també s'aplica en altres àrees de la geografia, incloent la hidrologia, on es monitoritzen els cossos d'aigua i s'estudien els canvis en la qualitat de l'aigua i els recursos hídrics; la geologia, per a la cartografia de formacions rocoses i l'anàlisi de riscos geològics; i la conservació del medi ambient, per a la protecció d'àrees naturals i la gestió de recursos naturals.

La capacitat de la teledetecció per proporcionar dades de gran abast espacial i temporal, combinada amb la seva versatilitat per captar informació en diverses bandes de l'espectre electromagnètic, la converteix en una eina indispensable per als geògrafs i professionals de disciplines relacionades. A través de la integració de dades de teledetecció amb altres fonts d'informació i l'ús de Sistemes d'Informació Geogràfica (SIG), és possible obtenir una comprensió més profunda i detallada del nostre món i les interaccions entre els seus components naturals i humans.

\hypertarget{avenuxe7os-tecnoluxf2gics}{%
\section{Avenços Tecnològics}\label{avenuxe7os-tecnoluxf2gics}}

Amb l'avanç de la tecnologia, els sensors de teledetecció han evolucionat significativament, oferint una major resolució, sensibilitat i capacitat per captar dades en diverses bandes de l'espectre electromagnètic. Això ha millorat la precisió i la varietat d'informació que es pot obtenir, obrint noves oportunitats per a l'exploració i l'anàlisi de la Terra i els seus recursos.

\hypertarget{desafiaments}{%
\section{Desafiaments}\label{desafiaments}}

Malgrat els seus nombrosos beneficis, la teledetecció també presenta desafiaments, incloent la interpretació correcta de les dades obtingudes, la integració de dades de diferents fonts i sensors, i la gestió de la gran quantitat de dades produïdes. A més, factors com la cobertura de núvols i la variabilitat atmosfèrica poden afectar la qualitat i l'ús de les dades de teledetecció.

\hypertarget{exercicis-1}{%
\section{Exercicis}\label{exercicis-1}}

\hypertarget{exercici-1-reflexiuxf3-sobre-la-teledetecciuxf3-en-al-bostre-curruxedculum}{%
\subsection*{\texorpdfstring{Exercici 1: Reflexió sobre la Teledetecció en al bostre \emph{currículum}}{Exercici 1: Reflexió sobre la Teledetecció en al bostre currículum}}\label{exercici-1-reflexiuxf3-sobre-la-teledetecciuxf3-en-al-bostre-curruxedculum}}
\addcontentsline{toc}{subsection}{Exercici 1: Reflexió sobre la Teledetecció en al bostre \emph{currículum}}

\begin{enumerate}
\def\labelenumi{\arabic{enumi}.}
\tightlist
\item
  \textbf{Reflexió Personal:} Penseu en les assignatures que heu cursat fins ara. Identifiqueu almenys dues on s'ha anomenat o utilitzat la teledetecció o productes derivats d'aquesta (per exemple, imatges de satèl·lit, mapes de vegetació, etc.). Descriviu breument com s'ha integrat la teledetecció en el contingut o les activitats d'aquestes assignatures.
\item
  \textbf{Anàlisi de Beneficis:} Reflexionau sobre com la inclusió de la teledetecció ha enriquit el vostre aprenentatge en aquestes assignatures. Considerau exemples específics de com les dades o les tècniques de teledetecció han millorat la vostra comprensió dels temes tractats.
\end{enumerate}

\hypertarget{exercici-2-aplicabilitat-de-la-teledetecciuxf3-en-diferents-disciplines}{%
\subsection*{Exercici 2: Aplicabilitat de la Teledetecció en diferents disciplines}\label{exercici-2-aplicabilitat-de-la-teledetecciuxf3-en-diferents-disciplines}}
\addcontentsline{toc}{subsection}{Exercici 2: Aplicabilitat de la Teledetecció en diferents disciplines}

\begin{enumerate}
\def\labelenumi{\arabic{enumi}.}
\tightlist
\item
  Assignatura Específica: Trieu una assignatura concreta (per exemple, Biogeografia, Climatologia, etc.) que encara ja heu cursat i que creieu que podria beneficiar-se de l'ús de la teledetecció. Argumenteu per què i cóm la teledetecció podria ser una eina útil en l'estudi d'aquesta disciplina.
\item
  Proposta de Projecte: Diseñau una proposta breu per a un projecte o activitat dins d'aquesta assignatura que utilitzi la teledetecció com a component clau. Detallau quins objectius podrien assolir-se amb l'ajuda de la teledetecció i quins tipus de dades o anàlisi serien necessaris.
\end{enumerate}

\hypertarget{exercici-3-integraciuxf3-transversal-de-la-teledetecciuxf3}{%
\subsection*{Exercici 3: Integració transversal de la Teledetecció}\label{exercici-3-integraciuxf3-transversal-de-la-teledetecciuxf3}}
\addcontentsline{toc}{subsection}{Exercici 3: Integració transversal de la Teledetecció}

\begin{enumerate}
\def\labelenumi{\arabic{enumi}.}
\tightlist
\item
  Visió Integradora: Penseu en un tema o problema global (per exemple, canvi climàtic, urbanització, conservació d'espècies, etc.) que requereixi un enfocament multidisciplinari. Expliqueu com la teledetecció podria servir com a eina de connexió entre diferents disciplines acadèmiques per abordar aquest tema.
\item
  Col·laboració Interdisciplinària: Proposeu un taller o seminari que reuneixi estudiants de diferents disciplines per treballar conjuntament en un projecte que utilitzi dades de teledetecció. Descriviu els objectius d'aquest taller, les activitats planificades i com les diferents perspectives disciplinàries poden enriquir l'anàlisi de les dades de teledetecció.
\end{enumerate}

\hypertarget{history}{%
\chapter{Història de la Teledetecció}\label{history}}

La teledetecció, tal com la coneixem avui, és el resultat d'una evolució fascinant de tècniques i tecnologies que es remunten a més d'un segle. Abans que el terme ``teledetecció'' fos formalment adoptat en els anys 1960, ja s'havien desenvolupat diverses metodologies per observar i analitzar la Terra des de la distància. Aquestes metodologies inicials van establir les bases sobre les quals la teledetecció moderna seria construïda, fusionant conceptes de fotografia, òptica, física i, més tard, tecnologia espacial.

\hypertarget{les-primeres-fotografies-auxe8ries-1859---1900s}{%
\section{Les primeres fotografies aèries (1859 - 1900s)}\label{les-primeres-fotografies-auxe8ries-1859---1900s}}

El naixement de la teledetecció pot rastrejar-se fins a l'adveniment de la fotografia aèria al segle XIX. Amb la introducció de globus, dirigibles i, eventualment, avions, els científics i els cartògrafs van començar a capturar les primeres imatges aèries de la Terra, proporcionant perspectives úniques que eren impossibles d'obtenir des del terra. Aquestes primeres fotografies aèries no només van servir per a propòsits militars i d'exploració, sinó que també van ajudar a avançar en el camp de la cartografia i van donar lloc a la ciència de la fotogrametria.

\begin{itemize}
\tightlist
\item
  Antigüitat: La càmera obscura (o caixa fosca) és un dispositiu que consistia en una caixa o habitació fosca amb un petit forat o lents a un costat. La llum que travessava aquest forat projectava una imatge invertida de l'exterior sobre la superfície oposada a l'interior de la caixa. Aquest principi va ser utilitzat per artistes i científics per a estudiar la perspectiva i la composició de la llum.
\item
  Segle XIX: Al principi del segle XIX, pioners com Joseph Nicéphore Niépce i Louis Daguerre van desenvolupar els primers processos fotogràfics que permetien fixar les imatges capturades amb una camera obscura sobre una superfície sensible a la llum. Niépce va realitzar la primera fotografia coneguda, ``Vista des de la Finestra a Le Gras'', al voltant de 1826 o 1827, utilitzant betum de Judea sobre una placa d'estany, un procés que requería hores d'exposició.
\item
  1839: Louis Daguerre va millorar la tecnologia fotogràfica, creant el \emph{daguerrotip}, un procés que reduïa significativament el temps d'exposició i millorava la qualitat de la imatge.
\item
  Plaques de Vidre i Pel·lícula Fotogràfica: El desenvolupament de plaques de vidre i, més tard, de pel·lícula fotogràfica flexible, van facilitar l'ús de la càmera fotogràfica, permetent la seva aplicació en camps com la teledetecció.
\item
  1783: Els germans Montgolfier van ser els pioners en volar un globus d'aire calent no tripulat.
\item
  1858-1859: Gaspard-Félix Tournachon, conegut com Nadar, obté la patent per a la fotografia aèria a França.
\item
  1860: James Wallace Black realitza la primera fotografia aèria conservada de Boston.
\item
  1888: George Eastman va jugar un paper clau en la transició de les plaques de vidre a la pel·lícula fotogràfica amb el llançament del primer rodet fotogràfic al 1888, facilitant així l'accés a la fotografia a un públic més ampli.
\item
  1903: Els germans Wright van realitzar el primer vol controlat d'un avió motoritzat.
\item
  1907: Julius Neubronner va patentar un mètode per equipar coloms amb càmeres lleugeres per prendre fotografies aèries.
\item
  1909: Wilbur Wright realitza la primera fotografia des d'un avió, marcant l'inici de la fotografia aèria moderna.
\end{itemize}

\hypertarget{la-primera-guerra-mundial-1914-1918}{%
\section{La Primera Guerra Mundial (1914-1918)}\label{la-primera-guerra-mundial-1914-1918}}

\begin{itemize}
\tightlist
\item
  1914: Els exèrcits comencen a utilitzar la fotografia aèria i globus per a l'observació dels moviments enemics de forma sistemàtica.
\item
  1915: J.T.C. Moore-Brabazon desenvolupa la primera càmera aèria de la història, dissenyada específicament per ser accionada des d'un avió. Aquest desenvolupament va ser crucial per al progrés de la fotogrametria aèria i va facilitar la confecció de cartografia basada en fotografia aèria. També va marcar l'inici dels primers llibres de fotointerpretació.
\item
  1916: S'introdueixen les primeres càmeres aèries específiques per a l'ús militar, millorant la recollida de dades.
\end{itemize}

\hypertarget{la-segona-guerra-mundial-1939-1945}{%
\section{La Segona Guerra Mundial (1939-1945)}\label{la-segona-guerra-mundial-1939-1945}}

Els avenços en teledetecció durant la Segona Guerra Mundial inclouen millores significatives en les tècniques fotogràfiques, la introducció de les primeres pel·lícules d'infraroig, la incorporació de nous sensors com el radar, avanços en els sistemes de comunicació, i progressos en la indústria aeronàutica. Aquests avenços van permetre una major precisió en el reconeixement fotogràfic, l'obtenció d'informació, la realització de bombardejos, i la producció de cartografia militar, la qual va ser d'immensa utilitat tant per a fins militars com civils, incloent el control de recursos naturals.

\begin{itemize}
\tightlist
\item
  1941: Introducció de la fotografia infraroja, permetent la detecció de camuflatges i la realització de missions de reconeixement en condicions de poca llum.
\item
  1943: Desenvolupament del radar d'obertura sintètica (SAR), millorant la capacitat de recollida de dades independentment de la llum solar o les condicions meteorològiques.
\end{itemize}

\hypertarget{la-guerra-freda-i-la-cursa-espacial-1950s-1960s}{%
\section{La Guerra Freda i la cursa espacial (1950s-1960s)}\label{la-guerra-freda-i-la-cursa-espacial-1950s-1960s}}

L'arribada de l'era espacial, marcada pel llançament del Sputnik per part de la Unió Soviètica el 1957, va representar un punt d'inflexió per a la teledetecció. La capacitat de situar sensors en òrbita al voltant de la Terra va proporcionar una plataforma inigualable per a l'observació continuada i global del planeta. Els satèl·lits, amb els seus avançats instruments de teledetecció, van començar a oferir dades detallades i sistemàtiques sobre la superfície terrestre, els oceans i l'atmosfera, revolucionant múltiples camps científics.

\begin{itemize}
\tightlist
\item
  1957: Llançament de Sputnik per la Unió Soviètica, el primer satèl·lit artificial de la Terra.
\item
  1960: TIROS-1, el primer satèl·lit meteorològic, és llançat pels EUA.
\item
  1965: Missió GEMINI-TITAN, que inclou les primeres fotografies formals d'ús geològic i meteorològic des de l'espai.
\end{itemize}

\hypertarget{lera-dels-satuxe8llits-civils-1972-present}{%
\section{L'Era dels satèl·lits civils (1972-present)}\label{lera-dels-satuxe8llits-civils-1972-present}}

\begin{itemize}
\tightlist
\item
  1972: Landsat-1 (anteriorment ERTS-1) proporciona les primeres imatges des de l'espai per a usos civils, incloent l'agricultura, la geologia i la hidrologia.
\item
  1978: Llançament del primer satèl·lit del Sistema de Posicionament Global (GPS), que, tot i que està més associat amb la navegació, també ha influït en la teledetecció per la millora de la geolocalització de les dades.
\item
  1986: Llançament del primer satèl·lit de la sèrie SPOT, proporcionant imatges amb major resolució i flexibilitat que Landsat.
\item
  1999: El satèl·lit Terra (EOS AM-1) de la NASA és llançat, portant instruments avançats per a l'estudi del sistema terrestre, incloent MODIS.
\end{itemize}

\hypertarget{exercicis-2}{%
\section{Exercicis}\label{exercicis-2}}

\hypertarget{exercici-1-comparaciuxf3-de-fites-histuxf2riques}{%
\subsection*{Exercici 1: Comparació de fites històriques}\label{exercici-1-comparaciuxf3-de-fites-histuxf2riques}}
\addcontentsline{toc}{subsection}{Exercici 1: Comparació de fites històriques}

\begin{itemize}
\tightlist
\item
  Objectiu: Avaluar i comparar dues fites específiques en la història de la teledetecció per determinar la seva importància relativa en el desenvolupament del camp.
\item
  Preguntes:

  \begin{enumerate}
  \def\labelenumi{\arabic{enumi}.}
  \tightlist
  \item
    Selecciona dues fites històriques clau en el desenvolupament de la teledetecció (per exemple, l'introducció de la fotografia aèria i el llançament del primer satèl·lit Landsat). Descriu breument cadascuna d'elles.
  \item
    Analitza com cadascuna d'aquestes fites va contribuir al progrés de la teledetecció. Considera factors com l'impacte tecnològic, l'aplicació pràctica, i la influència en futurs avenços.
  \item
    Argumenta quina de les dues fites consideres que ha tingut un major impacte en el desenvolupament de la teledetecció i per què.
  \item
    Quines lliçons es poden aprendre d'aquesta comparació sobre la naturalesa de la innovació en la teledetecció?
  \end{enumerate}
\end{itemize}

\hypertarget{exercici-2-classificaciuxf3-de-les-fites-muxe9s-importants}{%
\subsection*{Exercici 2: Classificació de les fites més importants}\label{exercici-2-classificaciuxf3-de-les-fites-muxe9s-importants}}
\addcontentsline{toc}{subsection}{Exercici 2: Classificació de les fites més importants}

\begin{itemize}
\tightlist
\item
  Objectiu: Crear una classificació de les fites més importants en la història de la teledetecció, basant-se en criteris específics com la innovació tecnològica, l'impacte social, o la contribució a la ciència.
\item
  Tasques:

  \begin{enumerate}
  \def\labelenumi{\arabic{enumi}.}
  \tightlist
  \item
    Escull cinc fites històriques significatives en el desenvolupament de la teledetecció.
  \item
    Defineix un conjunt de criteris per avaluar la importància d'aquestes fites (per exemple, innovació, impacte en la recerca, aplicacions pràctiques, etc.).
  \item
    Utilitza els criteris definits per crear una classificació ordenada d'aquestes fites, des de la més fins a la menys influent en el camp de la teledetecció.
  \item
    Justifica la teva classificació amb exemples concrets i argumenta per què algunes fites són més crítiques que altres per al desenvolupament de la teledetecció.
  \item
    Reflexiona sobre com aquesta classificació reflecteix l'evolució de la teledetecció i la seva importància en l'estudi del nostre planeta.
  \end{enumerate}
\end{itemize}

  \bibliography{../bib/book.bib,../bib/packages.bib}

\end{document}
